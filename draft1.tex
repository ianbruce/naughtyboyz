
\documentclass{amsart}

%     If your article includes graphics, uncomment this command.
\usepackage{graphicx}
\usepackage{amsmath,amsfonts,amssymb,amscd,amsthm,amsbsy,upref}
\usepackage[all]{xy}
\usepackage{amsmath}
\usepackage{mathrsfs}
\usepackage{paralist}
\usepackage{setspace}
\usepackage{graphicx}
\usepackage{tikz-cd}
\usepackage{pgfplots}
\usepackage{fancyhdr}
\usepackage{tikz}
\usepackage{pgfplots}
\usepackage{array}

\newtheorem{theorem}{Theorem}[section]
\newtheorem{lemma}[theorem]{Lemma}

\theoremstyle{definition}
\newtheorem{definition}[theorem]{Definition}
\newtheorem{example}[theorem]{Example}
\newtheorem{xca}[theorem]{Exercise}

\theoremstyle{remark}
\newtheorem{remark}[theorem]{Remark}

\numberwithin{equation}{section}

%    Absolute value notation
\newcommand{\abs}[1]{\lvert#1\rvert}

%    Blank box placeholder for figures (to avoid requiring any
%    particular graphics capabilities for printing this document).
\newcommand{\blankbox}[2]{%
  \parbox{\columnwidth}{\centering
%    Set fboxsep to 0 so that the actual size of the box will match the
%    given measurements more closely.
    \setlength{\fboxsep}{0pt}%
    \fbox{\raisebox{0pt}[#2]{\hspace{#1}}}%
  }%
}

\pgfplotsset{compat=1.13}

\begin{document}

\title{Assigning Scouts to Optimal Patrols}
\author{Phil Snyder}
\author{Ian Bruce}
\author{Julio Marco Pineda}

\begin{abstract}
The Scoutmaster wants to find an arrangement of 13 new scouts into two patrols. The cohesiveness and overall quality experience of a patrol is severely affected by adverse relationship and can be improved by existing preferences. Thus, by using anonymous data from scouts and parents, a brute force strategy was employed to find the optimal arrangement of scouts that avoids severe conflict and maximizes existing preferences. An arrangement of 6 and 7 scouts on one patrol respectively was found that satisfies the desired conditions. 
\end{abstract}
\maketitle
\section*{Problem Description}
Every year the Scoutmaster must assign new scouts to groups called patrols. In these patrols, scouts would plan and perform different scouting activities together in which they can form close friends that grow together. However, in recent years, the number of scouts have been increasing at a rate such that it is no longer logistically viable to place every new scout in a single patrol; multiple patrols are needed to organize activities better and to form proper bonds and community between the new recruits. Furthermore, the new scouts are in the fifth or sixth grade who have not fully developed their interpersonal skills. Many can be temperamental and awkward when dealing with adversity, and even some devastating conflicts occur where not only the two fighting scouts create animosity between each other, the rest of the troop alienates the destructive pair. If enough of these adverse interactions occur within a troop, many scouts would decide to quit and lose out in a great experience and community in the Boy Scouts.

The data the scoutmaster collects about adverse pairings and preferences are either privately given by the scouts themselves or by their parents. The possibly destructive interactions are given by the parents of the scouts.Previously, the Scoutmaster was able to determine the perfect patrols manually due to having smaller number of scouts. However, the number of scouts have been increasing yearly to the point that solving this grouping problem cannot be done by hand in a reasonable time. Therefore, we can help our community partner find a better means to solve his troop organization problem. Our goal for this project is to assign boy scouts in different patrols of appropriate size (of 6-8 scouts) such that we maximize the retention rate of the scouts in the program by avoiding severe conflicts while maximizing positive relationships between scouts.

We would like to answer the direct problem our community partner provided of arranging 15 scouts into two troops that avoids severe conflicts and promotes the positive relationships between the scouts. Furthermore, we want to determine if we can build a model to predict these patrol arrangements for any number of scouts so that the Scoutmaster can use this model for future years. Other than improving the overall experience of the scouts and alleviating the burden of the Scoutmaster, this model can possibly be applied to other situations where arranging a number of people of groups where interpersonal dynamics is necessary and important.

\section*{Simplifications}
Ultimately, our goal is to find groups for the scouts that will allow them to grow and be happy in the troop. If we really wanted to fully understand how a group can foster growth in an individual, we would want to consider the astronomical number of aspects about the personalities of each member and how those influence each of the other members given those other members’ astronomically complex personalities. Sociological and psychological literature would need to be studied extensively to learn about these aspects, and experts in the field would most likely be required to collect relevant data for each child, taking several weeks to conduct interviews. At the of the day, given our limited knowledge in this area, we settled with considering a subset of the sociological data between the scouts. We asked each scout which other scouts they liked and which ones they disliked. Now, this is obviously a giant simplification, but the difficulty of gathering more complex information and compiling said information would be too great for the expertise level of anyone involved in solving this problem. Also, this process of gathering data would have to be repeated each year for each new batch of new scouts, and would therefore be a burden on the Scoutmaster to collect a lot of data.

Another simplification is to assume that the quality of a group can be determined by the quality of the relationships between the unique pairs of the group - in other words, the whole is equal to the sum of the parts. For example, consider the case where scouts A, B and C individually like each other when they are alone with one other scout, but don’t like being in a group together. This epiphenomenon won’t be considered; in context of our model, the pairing of these scouts would be heavily favored. 

\section*{Mathematical Model}
This problem visually lends itself well to the notion of a mathematical graph. Since there is an inherent sense of relationships between individuals (how much a scout likes another scout), it is appropriate to consider the scouts as vertices in a graph with directed edges representing how much the “head” scout likes the “tail” scout. Our goal, using this language, is to find a partition of the vertices in the graph that satisfies some sort of objective function given size constraints of the partitions. In our problem at hand, we consider a complete graph with 13 vertices and edge weights being either 0, 1, or -1, corresponding to indifference, approval, or disapproval respectively of the “head” scout with respect to the “tail” scout. The Scoutmaster specifically wants two groups out of these 13, and with the size constraints for a patrol considered, our goal becomes finding 6 vertices to be in one partition and the other 7 being in another. The total number of partitions becomes 13 choose 6, which is equivalent to 1,716. This is certainly within the scope of considering total enumeration of the partitions as a reasonable solution to the problem, therefore we will proceed accordingly and implement the calculation using Python and its associated libraries to represent the data in a format that is convenient to work with.

Now, a natural question for us to consider is what objective function would be most appropriate to optimize. A natural answer to this would be to sum the edge weights of all edges that connect vertices that are within partitions. This would make sense if separating enemies is just as important as joining friends, but this just might not be the case. In terms of weighing these factors, we defer to Scoutmaster Bruce’s experience with how past patrols fared with friends and foes in the mix. Therefore, we consider a family of possible objective functions, namely giving real numbered weights to instances of approving relationships and disapproving relationships, and select a few of these objective functions to solve for. We then present the collection of results from different objective functions to the Scoutmaster for him to see which one seems to most reasonably match his preferences.
\section*{Solution of Mathematical Problem}
We represent our directed graph as a 13 by 13 similarity matrix, or, as might be more fitting for the problem, an “affinity” matrix. Like below:%
%%%%%%%%%%%%%%%%%%%%%%%%%%%%%%%%%%%%%%%%%%%%%%%%%%%%%%%%%%%%%%%%%%%%%%%%
\footnote{The scouts were enumerated like so: 1. Brandon, 2. Cameron, 3, Christian, 4. Colby, 5. Daniel, 6. Darwin, 7. Evan, 8. Jake, 9. Jordan, 10. Nathan, 11. Patrick, 12. Timmy, 13. Tommy.}%
%%%%%%%%%%%%%%%%%%%%%%%%%%%%%%%%%%%%%%%%%%%%%%%%%%%%%%%%%%%%%%%%%%%%%%%%
\includegraphics[scale=0.6]{solution1.JPG}

The NA value represents a hard constraint established by a scout’s parent (i.e., under no circumstances are these two to be in the same patrol). In our implementation we convert the NA value to $-\infty$. We designed an objective function that accepts two scouts who are assumed to be in different patrols and calculates the loss associated with such a placement. The loss is defined to be $L(x, y) = \text{max}(-2, x[y] + y[x])$ where $x[y]$ is $x$’s disposition towards $y$ and vise versa. For example, if $x$ dislikes $y$ and $y$ dislikes $x$, the loss is -2. If $x$ likes $y$ but $y$ dislikes $x$ the loss is 0. If $x$ and $y$ both like each other, the loss is 2. In mixed cases where the first scout is indifferent to another, but the other likes or dislikes the first scout, the loss is +1 or -1 respectively. We are trying to minimize the loss, so in applying our loss function over the entire dataset, our algorithm is equally averse to same-patrol mutual enmity as it is to dividing mutual friendship, half as averse to the indifference of one scout and the stronger inclination of another, and indifferent to polar opposite affinities within the same patrol. There is a question as to whether this is an accurate reflection of how individual scouts would view their own “loss” of any of the possible scenarios. As it turns out, our algorithm arrives at a very reasonable solution regardless. 
\section*{Results}
As mentioned earlier, the number of scouts we must partition is small enough to find an optimal solution by brute force. Our algorithm finds a partition of scouts into patrols of sizes 6 and 7 with a total loss of 0 as described in the table below:
\begin{table}[ht]
	\caption{}\label{eqtable}
	\renewcommand\arraystretch{1.5}
	\noindent\[
	\begin{array}{|c|c|}
	\hline
	\textbf{Patrol 1}&\textbf{Patrol 2}\\
	\hline
	\text{Brandon, Cameron, Colby,}&\text{Christian, Daniel, Jake,}\\
	\text{Darwin, Evan, Tommy}&\text{Jordan, Nathan, Patrick, Timmy}\\
	\hline
	\end{array}
	\]
\end{table}
\includegraphics[scale=0.65]{results1.JPG}
Of the undesirable outcomes of such a partition, we can spot a few:
\begin{enumerate}
	\item Tommy and Timmy both like each other and are in separate patrols. But if we move Timmy to Patrol 1 ($P1$), we break up the mutual friendship of Timmy and Nathan in $P2$. Suppose we moved both Timmy and Nathan to $P1$ and Brandon (who dislikes Nathan) to $P2$. But then Brandon loses two friends (Tommy and Cameron), gains one (Jordan), and now neither Nathan nor Timmy are in the same patrol as Daniel, whom they both like. Although, Brandon and Tommy find themselves in the awkward situation of having complete opposite sentiments towards each other. Perhaps this would be a worthwhile trade after all (and a larger, negative weight needs to be placed upon disfavor-favor relationships).
	
	\item As mentioned in the previous point, Tommy dislikes Brandon. Trading Tommy in $P1$ for Jordan in $P2$ seems promising, But we have overlooked the fact that Colby and Jordan are our NA pair, and cannot be in the same patrol under any circumstances. Trading Colby for Jordan is another option, but Colby will be missed by Darwin, Evan, Cameron and Tommy (In other words, the whole of P1 minus Brandon).
\end{enumerate}
On the whole, though, there are no obvious improvements that can be made to our algorithm’s partition. We find that this is an assignment that could have reasonably been arrived at by Scoutmaster Bruce (granted the group dynamics have not changed since the scouts were surveyed).  Furthermore, the solution took only 20 seconds to arrive at on a single core machine.

\section*{Improvements}
While our algorithm demonstrably works well on smaller groups of 13 people, we do not expect our brute force solution to be tractable on larger groups wherein greater than 2 partitions are required. As a possible scenario, consider a company of 150 people to be divided into fixed-size teams of 10 individuals each. There are $150! / (10!)^{15}$ possible partitions, a number proportional to $10^164$ - meaning we could instead use that computational time to enumerate the number of atoms in the universe… a tredecillion $(10^{78})$ times. Knowing this, perhaps it won’t surprise the reader that this particular graph partitioning problem in general graphs is NP-complete - though there do exist approximation algorithms.%
%%%%%%%%%%%%%%%%%%%%%%%%%%%%%%%%%%%%%%%%%%%%%%%%%%%%%%%%%%%%%%%%%%%%%%%%
\footnote{B. W. Kernighan, S. Lin, \textit{An Efficient Heuristic Procedure for Partitioning Graphs}. Bell System Technical Journal. \textbf{49} (1970), 291--307. doi: 10.1002/j.1538-7305.1970.tb01770.x}%
%%%%%%%%%%%%%%%%%%%%%%%%%%%%%%%%%%%%%%%%%%%%%%%%%%%%%%%%%%%%%%%%%%%%%%%%
As scout enrollment increases, it becomes necessary to consider scenarios where we have both a large number of scouts in need of assignment and multiple patrols to choose from. Our algorithm could be extended to use brute force to find a globally optimal solution when it is deemed computationally feasible and to use an approximating algorithm to find a locally optimal solution otherwise. 

\section*{Conclusions}
We initially struggled to decide on determining an appropriate objective function to partition the scouts to different groups. Follow up interviews with Scoutmaster Gene Bruce allowed us to decide on this function. We learned to rely on his expertise and intuition rather than trying to blindly decide on which types of relationships should be fostered and avoided to weight our objective function. Thus, we learned necessary communication skills and consideration of our community partner's needs.

With a feasibility check of how many solutions possible once the mathematical model was decided using graphs, a brute force strategy was the most accessible method to arrive at a solution. Thus we were able to develop a program that can partition scouts using this strategy. We learned to utilize tools we have at hand right away, and then we can consider other methods later on to reduce the computational time of our algorithm.

\section*{Acknowledgments}
We would like to thank Scoutmaster Gene Bruce for providing us the necessary data and being extremely patient and understanding when discussing the background and problem. We would also like to thank Professor Sarah Billey for the mentorship and guidance she provided throughout this project, and our TA Austin Tran for his comments and suggestions. 

\section*{Verification Statement}
(Once we receive our verification statement, we will place it here.)



\bibliographystyle{amsplain}
\begin{thebibliography}{10}

\bibitem {A} B. W. Kernighan, S. Lin, \textit{An Efficient Heuristic Procedure for Partitioning Graphs}. Bell System Technical Journal. \textbf{49} (1970), 291--307. doi: 10.1002/j.1538-7305.1970.tb01770.x

\end{thebibliography}

\end{document}

%------------------------------------------------------------------------------
% End of journal.tex
%------------------------------------------------------------------------------
